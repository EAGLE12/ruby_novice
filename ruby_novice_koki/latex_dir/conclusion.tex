
\section{結論}
同じ課題に対して,実際にarubaでのテストコードとtest::unitでのテストコードを書き,具体的に出力結果やコードを比較した.これにより,双方の良い点や問題点を抽出することができた.当初の開発目的が,「Ruby初心者が文法だけでなく,Rubyプログラミングにおける振舞いを身につけるための支援ソフトの開発」であった.この目的に合致させるためにはarubaが最適であった.なぜarubaなのか以下に簡単にまとめてみた.
\begin{description}
\item[textに忠実なcode] test::unitだとテストコードとスクリプトを同時に書かないといけないので,Ruby初心者にしては学習コストや間違えるリスクが大きくなる.またtext(たのしいRuby)で書かれているコードにreturnを付け加えなければならないというデメリットがある.それに比べてarubaだとtext(たのしいRuby)のコードをそのまま写すだけでよく,そのコードを実行するだけでテストをすることができる.

\end{description}\begin{description}
\item[個別テストの可能性] テスト環境としては, 環境変数RUBYNOVICE\_NAMEにディレクトリ名を入れるだけで,個人ごとにテストすることができる.また章ごとにテストコードを書いているので,各章ごとや各問題ごとにテストができ,1問ずつ確認しながらコードを書いていくことが可能である.

\end{description}
今後の課題としては,現段階でtextの7章までしかテストコードを書けていないので引き続き書くことであったり,慣れてきたらtextの問題だけでなく応用の問題もテストコードを書いていくことである.また問題にClassがあるコード(8章以降)は, 今まで通りコードを写すだけではテストできないので別のTDDフレームワークでのテストと比較して考える.

