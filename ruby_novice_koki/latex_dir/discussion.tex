
\section{考察}
\subsection{なぜaruba? (aruba vs test::unit)}
Cucumber,RSpec,Minitest (test::unit)のような人気のあるTDD/BDDフレームワークの中でもarubaを使用した理由は以下の通りである.
test:unitやarubaで書くとどうなるかを具体的に書いたコードを比べて示していく.

\subsubsection{test::unitで書いたテストコード}
たのしいRubyのテキストに記載されている問題で比較していきたいと思います.
テキストの最初の問題は,Hello, Rubyを出力するプログラムです.
\begin{quote}\begin{verbatim}
# list 1.1 (helloruby.rb)

print("Hello, Ruby.\n")
\end{verbatim}\end{quote}
まず,出力されるHello, Rubyをテストする場合のコードです.
\begin{lstlisting}[style=customRuby,basicstyle={\scriptsize\ttfamily}]
#helloruby.rb

def helloruby
  return "Hello, Ruby.\n"
end
\end{lstlisting}
\begin{itemize}
\item test::unitで書いたテストコード
\end{itemize}\begin{lstlisting}[style=customRuby,basicstyle={\scriptsize\ttfamily}]
require 'test/unit'
require './helloruby'

class Test_Sample < Test::Unit::TestCase
  def test_helloruby
    assert_equal("Hello, Ruby.\n",helloruby)
  end
end
print("Hello, ruby.\n")
\end{lstlisting}

テストコードの内容は以下の通りである.
Rubyで代表的なtest/unitというgemが提供されている.このプログラムの始め (require 'test/unit')で, test/unitを呼び出している.
Test::Unit::TestCaseを継承したクラスを用意し, test\_xxxというメソッドを定義するとそのメソッドがテストの実行対象になり,ここではそれぞれTest\_Sampleクラスとtest\_hellorubyメソッドがそれに該当する.
クラス名は大文字から始めるという規則があるので注意が必要である.またメソッド名は,必ずtest\_ から始めなくてはならない.ここでは単純にtest\_hellorubyとしている.実行してみると分かるが,
test\_ がないとちゃんと動いてくれない.
テストコードは,assert\_equal(期待値),(実行値)で実行結果を検証する. assert\_equalは,ふたつの引数をとり,第1引数は期待している結果で,第2引数はテストの対象である.両者が一致すればテストをパスし,一致しない場合はテストが失敗する.
補足ですが,test\_xxxというメソッドはクラス内に複数あっても構わない.また、1つのテストメソッド内にassert\_equalを複数書くのもOKである.
(とはいえ、原則として1テストメソッドにつき1アサーションとするのが望ましい)

このテストを実行すると以下のような出力になります.
\begin{lstlisting}[style=customRuby,basicstyle={\scriptsize\ttfamily}]
/Users/Koki/rubynovice/spec/test_unit/list1% ruby test_helloruby.rb 
Hello, ruby.
Loaded suite test_helloruby
Started.

Finished in 0.000982 seconds.

1 tests, 1 assertions, 0 failures, 0 errors, 0 pendings, 0 omissions, 0 notifications
100% passed

1018.33 tests/s, 1018.33 assertions/s
\end{lstlisting}
\subsubsection{test::unitでの問題点}

この場合だと初心者であるRubyの学習者がスクリプトとテストコードを同時に書かなければならない.学習者は,テストコードの書き方も学ぶ必要があるので,学習コストや間違えるリスクが大きくなる.
一番の問題点は,テキストを見ながら,その問題通りに書けないということである.先ほどの問題で説明すると,コードにreturnを付け加えなければならないことや,printメソッドはreturnできないので,テストするときは return "Hello, Ruby."と書き換えなければならない.
このようにtest::unitだとメソッドを書き換えなければならないことや,printメソッドをreturnで返すことができないというデメリットがある.
そこでarubaは, printをそのまま出力できテストが可能である.学習者がtext(たのしいRuby)を見ながら書いていけるというメリットがあるので学習コストや間違えるリスクを削減できる.
ここからは, 実際にarubaで書いたコードを元にして具体的に示していく.

\subsubsection{arubaで書いたテストコード}
先ほどと同じHello, Rubyを出力するプログラムをテストして比較する.
\begin{lstlisting}[style=customRuby,basicstyle={\scriptsize\ttfamily}]
# code.rb

def helloruby    
  print("Hello, Ruby.\n")
end
\end{lstlisting}\begin{lstlisting}[style=customRuby,basicstyle={\scriptsize\ttfamily}]
#ruby_novice.rb

require 'thor'                                                                         
require "code.rb"

module RubyNovice
  class CLI < Thor
    desc 'my_helloruby', 'print helloruby'
    def my_helloruby
      helloruby
    end
end
\end{lstlisting}
まずrequire で,thorとcode.rbを呼び出している.thorは,コマンドラインツールを作るためのgemである.
引数の受け渡しを簡潔に書くことができ,オプションのパースやUsage Message の表示など簡単に作成できる.

次にテストコードですが,arubaの場合printメソッドをreturnせずにそのままテストが可能になる.
下記がこの問題でのテストコードである.
\begin{lstlisting}[style=customRuby,basicstyle={\scriptsize\ttfamily}]
#ruby_novice_spec.rb

require 'spec_helper'

RSpec.describe 'ruby_novie command', type: :aruba do 
  context 'helloruby', type: :helloruby do
    before(:each) { run('ruby_novice my_helloruby') }
    expected = "Hello, Ruby."
    it { expect(last_command_started).to be_successfully_executed }
    it { expect(last_command_started).to have_output(expected) }
  end
end
\end{lstlisting}
テストコードの意味は次の通りである.
\begin{itemize}
\item run('ruby\_novice my\_helloruby') : ruby\_noviceのmy\_hellorubyを実行する.

\item expected = "Hello, Ruby." : 期待している結果. test::unitでいう第1引数である.

\item expect(last\_command\_started).to be\_successfully\_executed : status 0 で終了していることを確認. このコードでエラーなく終了したことを確認する.

\item expect(last\_command\_started).to have\_output(expected) : 出力がcontentsであることを確認, 正規表現も使用可能である.このコードで期待値=実際の値であるかを検証します.両者が一致すればテストをパスし,一致しない場合はテストが失敗する.

\end{itemize}
