\section{序論}
Rubyは本格的なオブジェクト指向プログラムが記述できる汎用性の高い日本発のオープンソースである.Rubyは初心者に分かり易く,プログラム教育にもスムーズに活用できるメリットがある\cite{1}. 西谷研究室に在籍している学生は,Rubyプログラミングを修得するために初心者向けの問題集を使って学習している.

ところが開発現場においては単に文法やプログラミングの書き方を知っているだけでは未熟で,より多くのスキルが要求される.典型的なものがバックアップに対するスキルである.バックアップをとるあるいはおいておくことはプログラミングの初心者に強調されるが,実際にバックアップのスキルを具体的に指示する指導は行われていない.現在のプログラミング環境においてはGithubがその標準となりつつある.Githubはバックアップだけでなく,進捗確認,バージョン管理やプルリクエストといった,チームによるプログラミングを促進するサービスが提供されている.

一方で,プログラミング開発の最先端の技法としてテスト駆動開発(Test Driven Development:TDD)が奨励されている.TDDでは仕様を満たすテストを書く(Red), テストが通るコードを書く(Green), コードを読みやすく直す(Refactoring)というステップでプログラミングを進めていくことを基本としている.それぞれの段階でなにに目標をおいて集中するかが明確になり,コード開発の効率が上がるとされている.

「初学者がこれらのスキルを自然と身につけることはできないか?」という問いに対する一つの答えとしてruby\_noviceを開発する.Ruby\_novcieが目指すものは,学習者自身が出力チェックできるようにしRubyプログラミングにおけるテスト実行に自然と慣れるような学習形態を目指している.さらに,進捗状況の管理や指導者からの添削をより容易におこなえるように改善するため,バージョン管理ソフトGithubを利用するシステム(ruby\_novice)を開発している.本研究はRuby初心者が文法だけでなく,プログラミングにおける振舞いを身につけるための支援ソフトを開発することを目的としている.

はじめに開発ソフトの仕様を2章で紹介する.さらに,開発したソフトの使用法を3章で述べる. 4章では, なぜarubaを使用したかの説明を述べている.

